% Template for ICASSP-2020 paper; to be used with:
%          spconf.sty  - ICASSP/ICIP LaTeX style file, and
%          IEEEbib.bst - IEEE bibliography style file.
% --------------------------------------------------------------------------
\documentclass{article}
\usepackage{spconf,amsmath,graphicx}

% Example definitions.
% --------------------
\def\x{{\mathbf x}}
\def\L{{\cal L}}

% Title.
% ------
\title{Bandwidth extension with no side information using dilated convolutional neural networks}
%
% Single address.
% ---------------
\name{Mathieu Lagrange, F\'elix Gontier \thanks{Work partially funded by ANR CENSE}}
\address{LS2N, CNRS, Centrale Nantes}
%
% For example:
% ------------
%\address{School\\
%	Department\\
%	Address}
%
% Two addresses (uncomment and modify for two-address case).
% ----------------------------------------------------------
%\twoauthors
%  {A. Author-one, B. Author-two\sthanks{Thanks to XYZ agency for funding.}}
%	{School A-B\\
%	Department A-B\\
%	Address A-B}
%  {C. Author-three, D. Author-four\sthanks{The fourth author performed the work
%	while at ...}}
%	{School C-D\\
%	Department C-D\\
%	Address C-D}
%
\begin{document}
%\ninept
%
\maketitle
%
\begin{abstract}

\end{abstract}
%
\begin{keywords}
One, two, three, four, five
\end{keywords}
%
\section{Introduction}
\label{sec:intro}

\section{Related work}
\label{sec:soa}

\cite{dietz2002spectral, ehret2004audio, ekstrand2002bandwidth, friedrich2007spectral, nagel2009harmonic}

\section{Model}
\label{sec:model}

\section{Experimental protocol}
\label{sec:protocol}

dataset

metric

baseline


\section{Experiments}
\label{sec:experiments}

ablation study

- dilation

- depth

- channels


\section{Discussion}
\label{sec:discussion}

\vfill\pagebreak


\bibliographystyle{IEEEbib}
\bibliography{strings,refs}

\end{document}
